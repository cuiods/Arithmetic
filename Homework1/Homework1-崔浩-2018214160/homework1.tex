\documentclass{article}
\title{Homework1}
\author{Hao Cui}
\usepackage{amsmath}
\usepackage{amssymb}
\usepackage{geometry}
\geometry{a4paper,scale=0.8}
\begin{document}
\maketitle
\section{Prove $2n+\Theta(n^2)=\Theta(n^2)$}
Let $f(n)\in\Theta(n^2)$, we can find $c_1,c_2,n_0$ such that $\forall n\ge n_0, 0\le c_1n^2 \le f(n) \le c_2n^2$, that is
$c_1n^2 + 2n \le f(n)+2n \le c_2n^2+2n$.\\
So we only need to prove $f(n)+2n\in \Theta(n^2)$, that means we need to find $c_3,c_4$ so that $\forall n\ge n_0, 0\le c_3n^2 \le f(n)+ 2n \le c_4n^2$.\\
Solve inequations:$
\left\{
\begin{aligned}
c_3n^2 \le c_1n^2 + 2n \\
c_4n^2 \ge c_2n^2+2n \\
\end{aligned}\right.$,
We can find $
\left\{
\begin{aligned}
c_3<c_1 \\
c_4>2/n_0 + c_2 \\
\end{aligned}\right. $ such that $\forall n\ge n_0, 0\le c_3n^2 \le f(n)+ 2n \le c_4n^2$.\\
Thus $f(n)+2n\in \Theta(n^2)$, and $2n+\Theta(n^2)=\Theta(n^2)$.
\section{Prove $\Theta(g(n))\cap o(g(n))=\emptyset$}
Assume $\Theta(g(n))\cap o(g(n))\neq\emptyset$, then exists $f(n) \in\Theta(g(n))$ and $f(n)\in o(g(n))$ ,
so $\exists c_1>0,c_2>0,n_0>0$ for that $\forall n \ge n_0,\forall c>0$,$
\left\{
\begin{aligned}
0\le c_1g(n) \le f(n) \le c_2g(n) \\
0\le f(n) < cg(n) \\
\end{aligned}\right.$,
let c = $\frac{c_1}{2}$, so $c_1g(n)\le f(n)<cg(n)=\frac{c_1}{2}g(n), c_1 < 0$.\\
This is a contradiction, thus $\Theta(g(n))\cap o(g(n))=\emptyset$.
\section{Prove $\Theta(g(n))\cup o(g(n))\neq O(g(n))$}
Let $
f(n)=\left\{
\begin{aligned}
g(n) \ when\ n\ is\ even  \\
0  \qquad \qquad otherwise\\
\end{aligned}\right.$,
$\exists c=1, n_0>0, s.t. \forall n\ge n_0,0\le f(n)\le cg(n)$,so $f(n)\in O(g(n))$.\\
If $0<c<1$ and n is even number, $f(n)>cg(n)$, so $f(n)\notin o(g(n))$.\\
If n is not even number, f(n)=0, there is not a $c_1>0$ so that $0\le c_1g(n) \le f(n)$, so $f(n)\notin \Theta(g(n))$.\\
Thus $f(n)\notin \Theta(g(n))\cup o(g(n))$.
Thus $\Theta(g(n))\cup o(g(n))\neq O(g(n))$.
\section{Prove $max(f(n),g(n))=\Theta(f(n)+g(n))$}
To show that $max(f(n),g(n))=\Theta(f(n)+g(n))$, we want to find constants $c_1,c_2,n_0>0$ such that $0 \le c_1(f(n)+g(n)) \le max(f(n),g(n)) \le c_2(f(n)+g(n))$ for all $n \ge n_0$.\\
Note that $max(f(n),g(n)) \le f(n)+g(n)$, we can find $c_2=1$ such that $ max(f(n),g(n)) \le c_2(f(n)+g(n))$.\\
Assume $f(n)\le g(n)$, and let $c_1=\frac{1}{2}$, we can find $\frac{1}{2}(f(n)+g(n)) \le f(n) \le g(n) = max(f(n),g(n))$.\\
So when $c_1=\frac{1}{2}, c_2=1$, we can find $n_0>0$ such that $0 \le c_1(f(n)+g(n)) \le max(f(n),g(n)) \le c_2(f(n)+g(n))$ for all $n \ge n_0$.\\
Thus $max(f(n),g(n))=\Theta(f(n)+g(n))$.
\section{Solve the recurrence $T(n)=2T(\sqrt{n})+1$}
Let $m=lgn$, then $T(2^m)=2T(2^\frac{m}{2})+1$.\\
Let $S(m)=T(2^m)$, then $S(m)=2S(\frac{m}{2})+1$.\\
According to the master method, $f(m)=1=O(m^{\log _{2}{2-1}})$, so that $S(m)=\Theta(m)$.\\
Thus $T(n)=T(2^m)=S(m)=\Theta(m)=\Theta(lgn)$.
\section{Solve the recurrence $nT(n)=(n-2)T(n-1)+2$}
$T(n)=1$ when $n \ge 2.$\\
Prove:\\
1. n=2,2T(2)=2, T(2)=1.\\
2. Assume T(k)=1,$k>2$, then (k+1)T(k+1)=(k-1)+2, then T(k+1)=1.\\
Thus $T(n)=1=\Theta(1)$.
\section{CLRS,pp61,3-3}
\subsection{Rank the following functions by order of growth}
Functions on the same line are in the same equivalence class.\\
\begin{center}
$2^{2^{n+1}}$\\
$2^{2^n}$\\
$(n+1)!$\\
$n!$\\
$e^n$\\
$n\cdot 2^n$\\
$2^n$\\
$(\frac{3}{2})^n$\\
$(lgn)^{lgn},n^{lglgn}$\\
$(lgn)!$\\
$n^3$\\
$n^2,4^{lgn}$\\
$nlgn,lg(n!)$\\
$n,2^{lgn}$\\
$\sqrt{2}^{lgn}$\\
$2^{\sqrt{2lgn}}$\\
$lg^2n$\\
$ln{n}$\\
$\sqrt{lgn}$\\
$lnlnn$\\
$2^{lg^*n}$\\
$lg^*n,lg^*(lgn)$\\
$lg(lg^*)n$\\
$n^{\frac{1}{lgn}},1$
\end{center}
\subsection{F(n) is neither $O(g_i(n))$ nor $\Theta(g_i(n))$}
$$
f(n)=\left\{
\begin{aligned}
2^{2^{2^{n+1}}} \ if\ n\ is\ even  \\
0  \qquad \ if\ n\ is\ odd\\
\end{aligned}\right.
$$
\end{document}





